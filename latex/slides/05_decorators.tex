% The Slide Definitions
\input{../templates/course_definitions}

% Author and Course information
\input{../templates/course_information}
\usepackage{comment}
% Presentation title
\title{Decorators}
\date{\today}


\begin{document}

\maketitle

\begin{frame}{Gliederung}
		\setbeamertemplate{section in toc}[sections numbered]
		\tableofcontents
\end{frame}


\section{Funktionen höherer Ordnung}
\begin{frame}{Funktionen sind Objekte}
	Funktionen sind Objekte mit speziellen Methoden.
	\lstinputlisting[firstline=1,lastline=13]{resources/06_decorators/higher_order.py}
\end{frame}
\begin{frame}{Folgen}
	\begin{itemize}
		\item Funktionen können Variablen zugewiesen werden,
		\lstinputlisting[lastline=8]{resources/06_decorators/functions.py}
		\item Sie können in Funktionen definiert werden,
		\lstinputlisting[firstline=11, lastline=19]{resources/06_decorators/functions.py}
	\end{itemize}
\end{frame}
\begin{frame}{Fakten über Funktionen}
	\begin{itemize}
		\item Sie können andere Funktionen zurückgeben,
		\lstinputlisting[firstline=22, lastline=29]{resources/06_decorators/functions.py}
	\end{itemize}
\end{frame}
\begin{frame}{Fakten über Funktionen}
	\begin{itemize}
		\item Sie können als Parameter mitgegeben werden
		\lstinputlisting[firstline=32]{resources/06_decorators/functions.py}
	\end{itemize}
\end{frame}

\subsection{Beispiele}
\begin{frame}{Beispiele - map}
	\alert{\texttt{map(function, iterable)}} wendet eine Funktion auf alle Elemente eines Iterators an.
	\lstinputlisting[firstline=14,lastline=30]{resources/06_decorators/higher_order.py}
\end{frame}
\begin{frame}{Beispiele - filter}
	\alert{\texttt{filter(function, iterable)}} gibt die Elemente eines Iterators zurück, für welche die Funktion \alert{\texttt{True}} zurückgibt.
	\lstinputlisting[firstline=32]{resources/06_decorators/higher_order.py}
\end{frame}


\section{Decorator}
\subsection{einfache Decorator}
\begin{frame}{einfache Decorator}
	Decorator sind Wrapper über existierende Funktionen. Dabei werden die zuvor genannten Eigenschaften verwendet.\\
	Eine Funktion, die eine weitere als Argument hat, erstellt eine neue Funktion.
	\lstinputlisting[lastline=9]{resources/06_decorators/decorator.py}
\end{frame}
\begin{frame}{einfache Decorator}
	\lstinputlisting[firstline=11,lastline=18]{resources/06_decorators/decorator.py}
\end{frame}

\begin{frame}[fragile]{einfache Decorator}
	Durch das \alert{\texttt{@} Symbol} lässt sich der Decorator wesentlich einfacher verwenden.
	\lstinputlisting[firstline=22, lastline=35]{resources/06_decorators/decorator.py}
	Es können auch mehrere Decorator übereinander geschrieben werden.
\end{frame}

\subsection{Decorator mit Argumenten}
\begin{frame}{Decorator mit Argumenten}
	Decorator erwarten Funktionen als Argumente. Aus diesem Grund kann man nicht einfach andere Argumente mitgeben, sondern man muss eine Funktion schreiben, die dann den Decorator erstellt.
	\lstinputlisting[firstline= 38]{resources/06_decorators/decorator.py}
\end{frame}
	
% nothing to do from here on
\end{document}
