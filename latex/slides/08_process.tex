% The Slide Definitions
\input{../templates/course_definitions}

% Author and Course information
\input{../templates/course_information}

% Presentation title
\title{Subprozesse in Python}
\date{\today}


\begin{document}

\maketitle

\begin{frame}{Gliederung}
        \begin{multicols}{2}
                \setbeamertemplate{section in toc}[sections numbered]
                \tableofcontents
        \end{multicols}
\end{frame}


\section{Grundlagen}
\begin{frame}[fragile]{Das Modul subprocess}
	Das Modul \alert{subprocess} erlaubt die Ausf\"uhrung externer Befehle und
	Skripte von einem Python Skript aus. Man kann sich die Funktionsweise \"ahnlich
	der eines Terminals vorstellen.
\end{frame}


\subsection{Eigenschaften}
\begin{frame}[fragile]{Eigenschaften}
	\begin{itemize}
		\item Subprozesse laufen \alert{asynchron}
		\item Sie laufen direkt auf dem System, nicht in einer Shell
		(wenn nicht anders festgelegt)
		\item Verf\"ugbare Programme h\"angen vom System ab, auf dem sie ausgef\"uhrt werden
		\item Der Aufruf ist allgemeing\"ultig, die Bibliothek verwandelt den
		Aufruf unter Windows in einen kompatiblen \texttt{CreateProcess()} String
	\end{itemize}
\end{frame}


\section{Konstanten}
\subsection{File Descriptoren}
\begin{frame}{File Descriptoren}
	\begin{description}
		\item[DEVNULL] Der systeminterne 'M\"ulleimer'
		\item[PIPE] Die Verbindung zwischen zwei Prozessen
		\item[STDOUT] Die Standardausgabe oder der laufende Prozess
	\end{description}
\end{frame}

\subsection{Exceptions}
\begin{frame}[fragile]{Exceptions}
	\begin{description}
		\item[\texttt{SubprocessError}] Der Standardfehler dieses Moduls
		\item[\texttt{TimeoutError}] Ein Timeout ist aufgetreten
		\item[\texttt{CalledProcessError}] Der Subprozess endete auf eine unerwartete Art
	\end{description}
\end{frame}


\section{Popen Klasse}
\begin{frame}[fragile]{Die Popen Klasse}
	Die \texttt{Popen} Klasse bildet die Basis des Moduls \texttt{subprocess}. \\
	Die Funktionssignatur sieht wie folgt aus:
	\lstinputlisting{resources/08_process/popen_call.py}
\end{frame}

\subsection{Wichtige Argumente}
\begin{frame}[fragile]{Einige wichtige Argumente}
	\begin{description}
		\item[\texttt{args}] Die aufzurufenden Argumente. Sollten vom Typ
			\alert{tuple} oder \alert{list} sein. Im Prinzip splittet man den
			Konsolenbefehl an den Leerzeichen: \\
			\texttt{ls -A *.md} entspricht \texttt{['ls', '-A', '*.md']}
		\item[\texttt{shell}] F\"uhrt den Befehl in einer Shell aus.
			Sollte \texttt{False} sein (Standard), sonst ist der Aufruf unsicher.
		\item[\texttt{stdout}] zusammen mit \texttt{stdin} und \texttt{stderr} die Input- und
			Output-Verbindungen des Subprozesses \\
			(hier sind \texttt{DEVNULL}, \texttt{PIPE} und \texttt{STDOUT} n\"utzlich)
		\item[\texttt{env}] Umgebungsvariablen des Kindprozesses. Standard ist ein
			Subset von \texttt{os.environ} (dem Python Prozess Environment)
		\item[\texttt{cwd}] Das Arbeitsverzeichnis des Subprozesses
	\end{description}
\end{frame}


\section{Popen Objekte}
\begin{frame}{Popen Objekte}
	Wenn Popen instanziiert wird, wird der darin enthaltene Prozess gestartet und
	das zur\"uckgegebene \texttt{Popen} Objekt enth\"alt Informationen \"uber den
	laufenden Prozess.
\end{frame}

\subsection{Informationen sammeln}
\begin{frame}[fragile]{Informationen sammeln}
	\begin{itemize}
		\item \texttt{process.args} \\
			Gibt die Argumente zur\"uck, mit denen der Prozess aufgerufen wurde.
		\item \texttt{obj.stdout}, \texttt{obj.stdin}, \texttt{obj.stderr} \\
			Input- und Output-Verbindungen, die beim Start gesetzt wurden
		\item \texttt{process.pid} \\
			Vom System zugewiesene \textit{Prozess ID}.
		\item \texttt{process.poll()} \\
			Pr\"uft, ob der Prozess beendet wurde. Gibt den \textit{R\"uckgabewert}
			des Prozesses zur\"uck oder \texttt{None}, wenn der Prozess noch l\"auft.
		\item \texttt{process.returncode} \\
			Der R\"uckgabewert von \texttt{process.poll()} (der R\"uckgabewert des Prozesses)
	\end{itemize}
\end{frame}

\subsection{Interaktion mit dem Prozess}
\begin{frame}[fragile]{Interaktion mit dem Prozess}
	\begin{itemize}
		\item \texttt{process.wait(timeout=None)} \\
			Wartet \textit{timeout} Sekunden auf die Terminierung des Prozesses
			(wartet unendlich lang, wenn timeout \textit{None} ist). \\
			Wirft nach Ablauf von \textit{timeout} eine \texttt{TimeoutExpired} Exception.
		\item \texttt{process.send\_signal(signal)} \\
			Sendet das Signal \textit{signal} an den Prozess (z.B. \texttt{SIGTERM}).
		\item \texttt{process.communicate(input=None, timeout=None)} \\
			Schreibt die Daten aus \textit{input} in den Standardinput (\texttt{stdin})
			des Prozesses (wenn stdin PIPE ist), wartet auf die Terminierung
			des Prozesses und liest die Daten, die der Prozess in \texttt{stdout}
			geschrieben hat (wenn stdout PIPE ist). \textit{timeout} funktioniert wie oben.
	\end{itemize}
\end{frame}

\begin{frame}[fragile]{Interaktion mit dem Prozess}
	\begin{itemize}
		\item \texttt{process.terminate()} \\
			Sendet ein Terminationssignal an den Prozess (\texttt{SIGTERM}).
		\item \texttt{process.kill()} \\
			Erzwingt die Beendigung des Prozesses (\texttt{SIGKILL}).
	\end{itemize}
\end{frame}

\subsection{Nutzung des Kontextmanagers}
\begin{frame}{Popen im Kontextmanager}
	Popen kann mit dem Kontextmanager verwendet werden (siehe \textit{File Handling}). \\
	Der Code daf\"ur w\"urde wie folgt aussehen:
	\lstinputlisting{resources/08_process/context_mgr.py}
\end{frame}

\section{N\"utzliche Funktionen}
\begin{frame}[fragile]{N\"utzliche Funktionen}
	\texttt{subprocess} enth\"alt einige Kurzfassungen f\"ur h\"aufig genutzte Arbeitsabl\"aufe.
	Intern rufen diese allerdings auch nur \texttt{Popen} auf.
\end{frame}

\begin{frame}[fragile]{run}
	Die wohl n\"utzlichste Funktion ist \texttt{subprocess.run()}:
	\lstinputlisting[lastline=2]{resources/08_process/defs.py}
	\begin{itemize}
		\item eingef\"uhrt in Python 3.5
		\item ruft durch \texttt{args} definierten Prozess auf
		\item wartet auf Beendigung des Prozesses (wenn \textit{timeout} \texttt{None} ist)
		\item gibt ein \texttt{CompletedProcess} Objekt zur\"uck
		\item \textbf{Beachte:} Einsatz eines unbenannten Aggregators
	\end{itemize}
\end{frame}

\begin{frame}[fragile]{CompletedProcess Objekt}
	Ist der Rückgabewert von \texttt{run()}, der zur\"uckgegeben wird, wenn der Prozess beendet wurde.
	Enth\"alt folgende Eigenschaften: \\[.25cm]
	\begin{description}
		\item[\texttt{args}] Argumente, mit denen der Prozess aufgerufen wurde.
		\item[\texttt{returncode}] R\"uckgabewert des Prozesses
		\item[\texttt{stdout}] Ausgabe des Prozesses
		\item[\texttt{stderr}] stderr Output des Prozesses \\[.5cm]
	\end{description}
	Au{\ss}erdem gibt es die Funktion \texttt{check\_returncode()}, die einen \texttt{CalledProcessError}
	beim Aufruf wirft, wenn der R\"uckgabewert nicht 0 ist.
\end{frame}

\begin{frame}[fragile]{call}
	\lstinputlisting[firstline=5, lastline=6]{resources/08_process/defs.py}
	\begin{itemize}
		\item Ruft einen Prozess auf, wartet auf Terminierung
		\item gibt Returncode des Prozesses zur\"uck \\[.5cm]
	\end{itemize}
	Diese Funktion macht das selbe, wie \\
	\hspace*{1.5cm}\texttt{run(\ldots).returncode}, \\
	nur dass \textit{check} und \textit{input} nicht unterst\"utzt werden.
\end{frame}

\begin{frame}[fragile]{check\_call}
	Argumente entsprechen den Argumenten von \texttt{call()}.
	\begin{itemize}
		\item Ruft Prozess auf, wartet auf Terminierung
		\item gibt nichts zur\"uck, wenn Ausf\"uhrung erfolgreich (Returncode == 0)
		\item wenn Aufruf nicht erfolgreich, wird \texttt{CalledProcessError} geworfen \\[.5cm]
	\end{itemize}
	Diese Funktion macht das selbe, wie \\
	\hspace*{1.5cm}\texttt{run(\ldots, check=True)}, \\
	nur dass \textit{input} nicht unterst\"utzt wird.
\end{frame}

\begin{frame}[fragile]{check\_output}
	\lstinputlisting[firstline=9]{resources/08_process/defs.py}
	\begin{itemize}
		\item f\"uhrt Kommando aus und gibt den Prozessoutput zur\"uck
		\item wirft ebenfalls \texttt{CalledProcessError}, wenn der Returncode nicht 0 ist \\[.5cm]
	\end{itemize}
	Diese Funktion macht das selbe, wie \\
	\hspace*{1.5cm}\texttt{run(\ldots, check=True, stdout=PIPE).stdout}
\end{frame}

% nothing to do from here on
\end{document}
